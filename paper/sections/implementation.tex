\documentclass[../manifest.tex]{subfiles}

\begin{document}
CodeStory is implemented as three components: (1) a Chrome extension responsible for collecting contextual information surrounding text copied from StackOverflow, (2) a backend receiving this information and persisting it in a database, and (3) an Atom package that
extends pasted snippets with a link to a page showing this information. Here we briefly describe the implementation details of each component and how they interact with each other.

\begin{table*}[t]
    \caption{Fields collected from StackOverflow upon copy}
    \label{tab:chrome-extension-fields}
    \centering
    \begin{threeparttable}
    \begin{tabular*}{\textwidth}{lll}
    \hline
    \textbf{Field Name} & \textbf{Description} \\
    \hline
    copiedFrom 					& Whether the snippet was copied from a question or an answer.\\
    type 						& Whether the copied snipped is code-only or a combination of code and text.\\
    originalSelection 			& Exact text that was selected in the browser upon copy.\\
    questionTitle 				& Title of the question of the current page.\\
    questionUrl      			& URL of the question (i.e. the URL of the current page).\\
    questionContent     		& Full content of the question as text.\\
    questionContentWithHtml     & Full content of the question preserving the HTML markup.\\
    questionVotes				& Number of votes the question received.\\
    answerUrl* 					& URL of the answer.\\
    answerContent*    			& Full content of the answer as text.\\
    answerContentWithHtml*		& Full content of the answer preserving the HTML markup.\\
    answerVotes*    			& Number of votes the answer received.\\
    accepted*					& Whether the answer was accepted.\\
    accessTime    				& Timestamp when the page was accessed.\\
    fullCodeSnippet    			& The full code snippet (only available if the selected code is part of a code snippet).\\
    \hline
    \end{tabular*}
    \begin{tablenotes}\footnotesize
        \item [*] Only available if copied from an answer.
    \end{tablenotes}
    \end{threeparttable}
\end{table*}

\subsection{Chrome Extension}
The CodeStory Chrome extension mainly consists of a Content Script that is executed whenever a user visits a StackOverflow page. An event listener for the browser's \textit{copy event} is added and called whenever content from the page is copied to the clipboard.

The listener first collects the required contextual information from the current page and saves it in a \textit{storyData} object whose fields are shown in Table ~\ref{tab:chrome-extension-fields}. This object is \texttt{POST}ed to the backend
with a unique ID, the hash of the object combined with the current time, whenever a user copies content from StackOverflow. We ensure that the backend accepts cross-origin requests so that the Chrome extension can simply use \texttt{Ajax} calls.

In order for our Atom package to be able to link to the contextual information, we include the ID in the clipboard content using the JavaScript clipboard API\footnote{https://developer.mozilla.org/en-US/docs/Web/API/ClipboardEvent}. The Atom package then uses the ID to form a URL which is included in the pasted content **Fig.**.

\subsection{Backend}
Our backend consists of a web server implemented in Node.js using the restify package and a redis database. The database stores the \textit{storyData} object using the ID mentioned in the previous section as the key. The web server provides three endpoints for creating and viewing the \textit{storyData} object. In particular, it provides two \texttt{REST} endpoints, \texttt{POST /codestory/rest} and \texttt{GET /codestory/rest/:id} for creating and retreiving the object, respectively, and a \texttt{GET /} endpoint for serving html files. An example of a CodeStory web page is shown in **Fig.**, which can be requested by using the URL in the CodeStory comment.

\subsection{Atom Package}
Atom is a popular, multi-platform, text editor that supports extending functionality with packages written in JavaScript. We created a simple package to provide enhanced pasting: if the clipboard content came from our Chrome extension then it will be inserted into the Atom editor with a comment above which includes a URL to the corresponding CodeStory web page. Otherwise, the standard paste operation is performed (i.e no comment is inserted). The correct commenting characters are used when pasting content into any file type supported by Atom. For example, \texttt{// View code story:\textit{URL}} would be inserted into a JavaScript file while \texttt{\# View code story:\textit{URL}} would be inserted into a Python file.

Maintaining expected paste behaviour was an important design choice that will affect developer adoption of the tool. We ens% This
% section provides a brief description of each part.
%
%
% The implementation of CodeStory consists of three parts that interact with each
% other. There is (1) a Chrome extension responsible for collecting contextual
% information from StackOverflow when snippets are copied, (2) a backend receiving
% this information and persisting it in a database and (3) an Atom package that
% extends pasted snippets with a link to a page showing this information. This
% section provides a brief description of each part.ured that only clipboard content ending with the special tag, \texttt{CodeStory:\textit{<HASH>}}, would alter the default paste behaviour. Further, we set the default key-binding to be CTRL+SHIFT+V but the user can change the binding to override the default paste command by setting the binding to CTRL+V. Alternatively, the paste operation can be invoked from the packages menu in Atom.

\end{document}
