\documentclass[../manifest.tex]{subfiles}

\begin{document}
Software is the result of decisions made based on information obtained from many sources. While software implicitly captures the results of these decisions, their rationale evaporates unless some additonal process is used to capture it. In practice, documenting knowlegde is often considered a resource-intensive process without tangible, short-term gains, so often it is skipped or performed inadaquately \cite{OZ2008,NH2007}. However, we found that it is exactly this rationale that helps developers provide effective code reviews and could help with other comprehension tasks.

Developers do not program in isolation: they use resources on the internet and colloborate with other people to accomplish a programming task. Communication tools such as email, instant messaging and web sites like StackOverflow make it easy for our tool to capture knowledge as it is transferred between individuals. This is due to the fact that these forms of communication are already in a text-based representation and, more importantly, follow a typically concise format of question-discussion-answer. We suggest that by linking these knowledge transfers to the sections of code that prompted the inquiry, significant value can be added in the form of increased future comprehension.

To exploit this, we created a tool called CodeStory that captures the rationale behind this question-discussion-answer communication and links it to the resuling code. It does this by inserting a comment with a hyperlink pointing to a web page that describes when and where the information came from and what the original question was; i.e. the story behind the code.

CodeStory currently supports the popular developer Q\&A site StackOverflow as the knowledge transfer medium. On this site, developers post technical questions about a problem they have encountered and then, after reviewing several possible solutions contributed by other developers, they choose the one that best fits their requirements. We take the action of a developer copying some or all of the content of an answer, typically a code snippet, as indicating their choice. At this point we collect additional information that forms the code story for the copied snippet and include a link to it when the snippet is pasted into code.

The contributions of this paper are as follows:
\begin{itemize}
  \item A simple approach for capturing text-based collaborative rationale among
  developers as it is transfered from one developer to another in question-discussion-answer
  type interactions. This can include communications using email, instant messaging, and
  websites. The approach is also appropriate when the developer has a question
  that can be answered by consulting digital documentation.
  \item A prototype tool that implements this approach. It captures contextual
  information surrounding snippets copied from StackOverflow using a Google Chrome extension and a server-side database. An Atom\footnote{https://atom.io} package lets developers paste the snippet with a hyperlink pointing to a web page showing the corresponding code story.
\end{itemize}

% Section 2 presents related work. In section 3 we discuss our design choices and rationale when implementing CodeStory.
%
% We discussion future work in section 7 and conclude in section 8.
%
% 3-approach
% 4-implementation
% 5-evaluation
% 6-discussion
% 7-future Work
% 8-conclusion

\end{document}
