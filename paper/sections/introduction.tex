\documentclass[../manifest.tex]{subfiles}

\begin{document}
A program is the result of decisions made based on information obtained from a
variety of sources. Knowledge of these decisions can quickly evaporate if there
is no process for capturing it. In practice, documenting knowlegde is often
considered a resource-intensive process without tangible, short-term gains, so
often it is skipped or performed inadaquately\cite{OZ:2008,NH:2007}. However,
this information is useful in program comprehension tasks. In particular, we found
there is strong industry support in  code
review.



approach  **introduce the information as a code story**

The contributions of this paper are as follows:
\begin{itemize}
  \item A simple approach for capturing text-based collaborative reasoning among
  developers. This can include communications using email, instant messaging,
  websites and references.
  \item A prototype tool that implements this approach. It captures contextual
  information surrounding snippets copied from StackOverflow and stores it in a
  database using a Google Chrome extension. An Atom\footnote{https://atom.io}
  package lets developers paste the snippet with a hyperlink to the code story.
\end{itemize}

Section 2 presents related work, ...

% Capturing this information and making it available during
% comprehension tasks like code reviews, maintenance, and introducing a codebase
% to new developers, would make these tasks eaiser. Unfortunately, this information
% is both hard to capture and hard to present in a meaningful way.
%
% Many tools focus on extracting the information
%
%  While this information helps code reviewers in understanding
% However, this information canHowever, this information can be useful in program comprehension tasks be useful in program comprehension tasks
%
%
%  - What is the information






\end{document}
