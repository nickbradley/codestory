\documentclass[../manifest.tex]{subfiles}

\begin{document}

Anything that does not directly result in usable code is an extra cost to development companies. However, capturing decision rationale is important for future development. Thus, we want to minimize the upfront cost of the capture process while simultaneously maximizing the future benefits of this captured information. Taking these constraints into consideration we decided on four key design decision criteria for CodeStory.

\textbf{Minimize impact to source code.} Developers do not want to be distracted by unnecessary or verbose comments in their code. Because our tool hooks into the copy-paste operation, it can capture a great deal of information but including all of it as comments in the code is not desirable. After trying several different levels of comment verbosity, we concluded that a single-line comment would be the most appropriate even though it would require extra steps to see the full code story. Extending our tool to support displaying the story in the IDE would help mitigate this. Not including any comment would be ideal but that presents many challenges for linking the story with its corresponding code.

\textbf{Transparently integrate into existing workflows.} A tool must be used if it is to provide any benefit. Since CodeStory captures information useful to future developers, it is imperative that it not require any extra effort on the part of the current developer. Hooking into the standard copy-paste operation was an obvious way to capture information that the user was interested in bringing into their code without requiring any explicit actions.

Our tool consists of three components, each of which is easily modifiable, making it easy to extend CodeStory to support user's existing tool sets. CodeStory automatically captures contextual information when a copy command is invoked in a supported source program and pasting with the code story hyperlink can be done by a user defined keybinding, including the default paste keybinding. The code story is stored separately from the code: the insertion of a standard one-line comment is the only modification to the code.

\textbf{Capture the right contextual information.} The goal of CodeStory is to capture and describe the rationale behind
decisions made in code without requiring further action from the developer. We had to decide what information would be most pertient to this task: capturing too little would have made it hard to understand the rationale while too much would have obfuscated the key points. To help us decide, we used a pilot survey which we describe in section \ref{eval-survey}.


\textbf{Present the captured information in a meaningful way.} To allow developers to benefit from a code story during program comprehension tasks it is important that the story be close to the target code, viewable in any tool, and be easy to interpret. As mentioned before, CodeStory inserts a hyperlink directly above the pasted code. By default, the code story can be served as an HTML web page so that it can be viewed in most IDEs and in any web browser. It can also be served in other formats such as JSON and XML to make it viewable in custom applications. The code story summarizes the raw captured information in natural language making it easier to read and understand.

\end{document}
