\documentclass[../manifest.tex]{subfiles}

\begin{document}

In this paper we presented an approach for capturing decision rationale and provided a protoype tool, CodeStory, that implements this approach. Code implicitly captures the result of decisions but these decisions are the result of question-discussion-answer type discussions. CodeStory enables developers to transparently collect this rationale by capturing contextual text surrounding copied text. We found that by making this information available to code reviewers, the quality of reviews was generally better. Currently, the prototype only works with StackOverflow but the approach applies more generally to other common communication applications like email and instant messaging. We also have yet to see if this applies to other tasks requiring program comprehension tasks like bug fixes.

We developed our approach by conducting a pilot survey to establish what information on StackOverflow would provide the most utility to code reviewers presented with unknown and otherwise undocumented code. We then evaluated the usefulness of the captured rationale by having participants each perform two code reviews, one with a code story and one without. The participant's feedback showed that they found the code story helpful in performing the code review task.

Overall, we believe that capturing the rationale for question-discussion-answer type communications is important for improving software quality and that CodeStory provides a minimally distruptive way to do so.

\end{document}
