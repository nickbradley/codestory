\documentclass[../manifest.tex]{subfiles}

\begin{document}
The results of our study show that capturing and displaying contextual information makes code reviews eaiser. In order to make our study coherent and to keep it to a reasonable length for our participants, we decided to forego evaluating the  CodeStory tool directly. However, there are several interesting questions that would help us improve CodeStory. Some of these include: Should more information be shown in comments? Does CodeStory fit into developers' existing workflows? Would developers keep the tool installed after a week of first using it?

We would also like to determine if the code story has different utility for different comprehension tasks. It would be very interesting to know, for example, whether bugs can be found and/or fixed more easily in annotated codebases. Or, if new features are incorporated into the software in a more consistent manner or in less time because developers have a chain of rationale to follow.

The prototype can be extended to support more sources like email clients and instant messaging applications which are common tools used by development teams. Currently, a user needs to open the code story in a web browser which can break their workflow. Extending our IDE package to support displaying the code story in the IDE would make it easier to take advantage of the code story. Adding support for other popular tools that are used for specific comprehension tasks is also important. For example, in the code review tasks given to the participants, we wanted to show the code story in a popup window in Github when the reviewer hovers over the link.

Finally, we would like to evaluate the effectiveness of our presentation of the code story.

\end{document}



  % - does it fit into developer's workflows? Does it interfer with other tools?
  % - Would they use the tool? Would they make our tool the default for pasting?
  % - Recommendatios for improvements


  %  work  so developers don't have to switch apps (which will no doubt break their workflow we've been so careful to avoid interrupting). On a similar note, you would need to develop the plugin for different IDEs (can't evaluate impact on workflow if the developer has to use an unfamiliar IDE).
